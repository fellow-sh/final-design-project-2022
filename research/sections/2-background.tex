\section*{Background}

Deicers are predominantly composed of sodium chloride, or salt.
A sodium chloride solution with water has a higher freezing point than a pure water liquid.
Hence, when it is dissolved into the ice on roads, the ice melts into water.
Some consumer-grade deicers also contain calcium chloride, which creates an exothermic reaction upon contact with water.
This makes it quicker at deicing than sodium chloride, and an appropriately balanced mixture of both compounds can result in an effective and environmentally safer mixture since less overall substance is used.
However, on larger scales, using a calcium chloride--sodium chloride mix is uneconomical so calcium chloride is not used for city-wide deicing.
The same can be said for other deicing compounds, either due to difficult sourcing or extraction \parencite{ARF15}.

Alternatively, some cities will try to increase the traction on roads using sand.
It is easy to source in most places while being cheap and abundant.
However, the sand is often pushed off the roads in the spring, creating lasting sediment that can clog in waterways and basins, requiring huge investment to clean up and maintain \parencite{EPA20}.
The use of biodegradable deicing mixtures is therefore appealing, since they should have a lower environmental impact and could potentially utilize agricultural byproducts.
Bain, Gamayo, and Morant explored this idea by testing several agricultural byproducts that were local to Manitoba, and found that hemp hurd was particularly effective in supplementing sodium chloride for deicing \parencite{KAV21}.
Specifically, the hemp hurd was effective in absorbing the escaping moisture while also providing a degree of traction.
Additionally, considering the growing hemp industry in Canada, the use of hemp hurd can be seen as an effort to increase the utility output of hemp plants and reduce waste in production.
As such, the mixture was presented to Winnipeg's city council as a solution to reducing the amount of salt used on roads every winter.
