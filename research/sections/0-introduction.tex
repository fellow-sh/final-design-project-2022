\section*{Introduction}

Road salt is commonly used during the winter, preventing both pedestrians and vehicles to safely move over frozen roads.
It is estimated that, of the 54,000,000 metric tons of salt that was consumed in 2021, 42\% of it was used for road deicing \parencite{USG22}.
Such major use of salt has had a significant impact on local ecosystems by increasing the salinity of waterstreams, threatening the state of the surrounding environment \parencite{KGW20}.
Notably, excess salt can enter lakes and rivers, significantly reducing their biodiversity \parencite{HKM22}.
Additionally, salt can also enter fresh drinking water, increasing the salt content such that it meets up to 33\% of an adults recommended daily intake \parencite{CGR22}.
Road salt also leads to damages on infrastructure, costing an estimated \$5 billion in repairs \parencite{EPA20}.

The affects of road deicing have been known for some time now, and many municipalities have investigated reducing the usage of salt.
A common tactic is to pre-wet road salt such that vehicle tires do not throw salt off roads \parencites{ZAS20}{UFK17}.
Other techniques include using a salt-sand mixture to increase road abrasion, creating biodegradable mixtures, and making porous pavement parking lots \parencite{EPA20}.
More recently have biodegradable mixtures been explored as suitable alternatives to salt for road deicing.
One such mixture suggests combining salt with hemp hurd, proposed originally by high school students in Manitoba \parencite{KAV21}.
This mixture is significant compared to other alternatives because it utilizes and promotes the locally growing hemp industry within the province.

This study aims to find the ideal mixture of hemp hurd and salt that works as an affective alternative to traditional road salt.
